\documentclass[12pt]{article}
 
\usepackage[margin=1in]{geometry} 
\usepackage{amsmath,amsthm,amssymb,graphicx,mathtools,tikz,hyperref}
\usetikzlibrary{positioning}
\newcommand{\n}{\mathbb{N}}
\newcommand{\z}{\mathbb{Z}}
\newcommand{\q}{\mathbb{Q}}
\newcommand{\cx}{\mathbb{C}}
\newcommand{\real}{\mathbb{R}}
\newcommand{\field}{\mathbb{F}}
\newcommand{\ita}[1]{\textit{#1}}
\newcommand{\com}[2]{#1\backslash#2}
\newcommand{\oneton}{\{1,2,3,...,n\}}
\newcommand{\abs}[1]{|#1|}
\newcommand\idea[1]{\begin{gather*}#1\end{gather*}}
\newcommand\ef{\ita{f} }
\newcommand\eff{\ita{f}}
\newcommand\proofs[1]{\begin{proof}#1\end{proof}}
\newcommand\inv[1]{#1^{-1}}
\newcommand\set[1]{\{#1\}}
\newcommand\en{\ita{n }}
\newcommand{\vbrack}[1]{\langle #1\realangle}

\realenewcommand\qedsymbol{$\blacksquare$}

\newtheorem{thm}{Theorem}[section]
\newtheorem{defn}{Definition}[thm]
\newtheorem*{remark}{Remark}
\newtheorem{prop}{Proposition}[thm]
\newtheorem{example}{Example}[thm]
\newtheorem{lemma}{Lemma}[thm]
\newtheorem{exercise}{Exercise}[thm]
\hypersetup{
  colorlinks,
  linkcolor=blue
}
\begin{document}
\date{}

 
\title{MAT 4144 Lie Groups}
\author{Notes by: David Draguta
  \\University of Ottawa\\ Prof: Dr. Tanya Schmah } 
 
\maketitle
\section{Introduction}

\begin{defn}
  A \textbf{topological space} is an ordered pair $(X, \tau)$, where $X$ is a set and $\tau$ is a collection of subsets of $X$, satisfying the following axioms.
  \begin{enumerate}
  \item The empty set and $X$ itself belong to $\tau$.
  \item Any arbitrary (finite and infinite) union of members of $\tau$ still belongs to $\tau$.
  \item The intersection of any finite number of members of $\tau$ still belongs to $\tau$.
  \end{enumerate}
\end{defn}

\begin{defn}
  A function $f: X \to Y$ between two topological spaces is a \textbf{homeomorphism} if it has the following properties:
  \begin{itemize}
  \item f is a bijection.
  \item f is continuous
  \item the inverse function $\inv{f}$ is continuous 
  \end{itemize}
  If such a function exists, $X$ and $Y$ are said to be \textbf{homeomorphic}.
\end{defn}

\begin{defn}
  A \textbf{cover} of a topological space $X$ is a collection $C = \set{U_{\alpha} : \alpha \in A}$ such that
  \begin{center}
    $X \subseteq \bigcup\limits_{\alpha \in A} U_{\alpha}$
  \end{center}
\end{defn}
\begin{defn}
  A \textbf{base} or \textbf{basis} for the topology $\tau$ of a topological space $(X,\tau)$ is a collection $B$ of open subsets of $X$ satisfying the following properties.
  \begin{enumerate}
  \item The base elements cover $X$
  \item Let $B_1, B_2$ be base elements and let $I$ be their intersection. Then for each $x \in I$, there is a base element $B_3$ containing x such that $B_3$ is a subset of $I$.
  \end{enumerate}
\end{defn}

\begin{defn}
  A topological space $X$ is \textbf{second-countable} if its topology has a countable base.
\end{defn}

\begin{defn}
  If $X$ is a topological space and $p$ is a point in $X$, a \textbf{neighbourhood}
  of $p$ is a subset $V$ of $X$ that includes an open set $U$ containing p such that $ p \in U \subseteq V$
\end{defn}
\begin{remark}
  In a topological space $(X, \tau)$, the open subsets are the elements of $\tau$.
\end{remark}
\begin{defn}
  Points x and y in a topological space $X$ can be \textbf{separated by neighborhoods}
  if there exists a neighborhood $U$ of $x$ and $V$ of $y$ such that $U \cap V = \emptyset$
\end{defn}
\begin{defn}
  $X$ is a \textbf{Hausdorff space} if all distinct points in $X$ can be separated by neighborhoods.
\end{defn}
\begin{defn}
  A topological space $X$  is called \textbf{locally Euclidean} if there is a non-negative integer $n$ such that every point in $X$ has a neighbourhood wihch is homeomorphic to $\real^n$.
\end{defn}

\begin{defn}
  A \textbf{topological manifold} is a locally Euclidean Hausdorff space. 
\end{defn}
\begin{defn}
  Let $M$  be a topological space. A \textbf{chart} (U, $\phi$) on $M$ consists of an open subset $U$ of M, and a homeomorphism $\varphi$ from $U$ to an open subset of some Euclidean space $\real^n$.
\end{defn}

\begin{defn}
  A differentiable  \textbf{atlas}  is a collection of charts $\set{\varphi_{\alpha} : U_{\alpha} \to  \real^n}_{\alpha \in A}$ such that $\set{U_{\alpha}}_{\alpha \in A}$  covers $M$, and such  that for all  $\alpha$
  and $\beta$ in $A$, the transition map $\varphi_{\alpha} \circ \inv{\varphi_{\beta}}$ is a smooth ($C^{\infty}$) map.
\end{defn}

%make diagram of the transition map of two charts.

\begin{defn}
  Given a differentiable atlas on a topological  space, one says that a chart is \textbf{differentiably compatible} with the atlas, if the  inclusion of the chart into the collection of charts that is the atlas results in another differentiable atlas.
\end{defn}

\begin{defn}
  A differential atlas determines a \textbf{maximal differentiable atlas}, consisting of all charts which are differentiably compatible with the given atlas.
\end{defn}

\begin{defn}
  A \textbf{differentiable manifold} is a Hausdorff and  second countable topological space  $M$, together  with a maximal differentiable  atlas on $M$.
\end{defn}

\begin{defn}
  A \textbf{real lie group} is a group that is also a finite-dimensional real smooth manifold, in which the group operations of multiplication and inversion are smooth maps. Smoothness of the group multplication
\begin{center}
  $\mu : G \times G \to G$,  $\mu(x,y)=xy$
\end{center}
  means that $\mu$ is a smooth mapping of the product manifold $G \times G$ into $G$.
\end{defn}

\begin{example}
  The following are lie groups. It is an exercise to show that they are all isomorphic.
\begin{itemize}
\item
  $S_1 = \set{(cos \theta, sin \theta) \in \real^2 : \theta \in [0,2\pi)}$, with operation
  \begin{center}
    $(cos \theta_1, sin \theta_1) * (cos \theta_2, sin \theta_2) = (cos (\theta_1 + \theta_2), sin (\theta_1 + \theta_2))$.
  \end{center}
 
\item
    $S_{\cx}^1 = \set{z \in \cx : \abs{z} = 1}$, with operation complex multiplication.

\item
    $SO(2) = \set{ \begin{pmatrix}
      cos \theta & - sin \theta\\
      sin \theta & cos \theta 
    \end{pmatrix} : \theta \in \real}$
  , with operation matrix multiplication.
\end{itemize}
What's more, these are actually isomorphic as lie groups, which we shall see later. 
\end{example}

\begin{example}
  Orthogonal groups are lie groups too.
  \begin{itemize}
  \item
    $O(n) = \set{A \in M_n(\real) : A^TA = I}$
  \item
    $SO(n) = \set{A \in O(n) : det A = 1}$
  \end{itemize}
\end{example}
\end{document}

\begin{example}
\end{example}







